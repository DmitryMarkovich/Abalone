%%%%%%%%%%%%%%%%%%%% Preamble %%%%%%%%%%%%%%%%%%%%
\documentclass[10pt, paper=a4]{article}
\usepackage{amssymb,amsfonts,amsmath,latexsym,amsthm, mathtools} %mathtext,
\usepackage{booktabs}
\usepackage{multirow}
\usepackage{graphicx}
\usepackage{listings}
\usepackage{chngpage}
\usepackage{cprotect}
\usepackage[font=footnotesize, labelsep=period]{caption}
\usepackage{cite}
\usepackage[scale=0.925]{geometry}
\graphicspath{{images/}}
\usepackage[pdftex,unicode,colorlinks, citecolor=blue,
  filecolor=black, linkcolor=blue, urlcolor=blue]{hyperref}
\usepackage[figure,table]{hypcap}
%%%%%%%%%%%%%%%%%%%% Document %%%%%%%%%%%%%%%%%%%%
\begin{document}
%%%%%%%%%%%%%%%%%%%% Title page %%%%%%%%%%%%%%%%%%%%
\title{Report 03 --- Clustering, association mining and outlier detection in
  abalones}

\author{Dmitriy Markovich (s146577) and Julian Lemos Vinasco (s150959)}

\date{}

\maketitle

\begin{abstract}
Objective: The objective of this third and final report is to apply the methods
you have learned in the third section of the course on ``Unsupervised learning:
Clustering and density estimation'' in order to cluster your data, mine for
associations as well as detect if there may be outliers in your data.
\end{abstract}

%%%%%%%%%%%%%%%%%%%% Clustering
\section{Clustering}
\label{sec:clustering}
In this part of the report you should attempt to cluster your data and evaluate
how well your clustering reflects the labeled information. If your data is a
regression problem define two or more classes by dividing your output into
intervals defining two or more classes as you did in report 2.

\begin{enumerate}
  \item Cluster your data by the Gaussian Mixture Model (GMM) and use cross-
    validation to estimate the number of components in the GMM. Try to interpret
    the extracted cluster centers.
  \item Perform a hierarchical clustering of your data using a suitable
    dissimilarity measure and linkage function. Try to interpret the results of
    the hierarchical clustering.
  \item Evaluate the quality of the clustering in terms of your label
    information for the GMM as well as for the hierarchical clustering where the
    cut-off is set at the same number of clusters as estimated by the GMM.
\end{enumerate}

%%%%%%%%%%%%%%%%%%%% Association mining
\section{Association mining}
\label{sec:association}
In this part of the report you are to investigate if there are associations
among your attributes based on association mining. In order to do so you will
need to make your data binary, see also exercise 10. (For categoric variables
you can use the one-out-of-K coding format). You will need to save the binarized
data into a text file that can be analyzed by the Apriori algorithm.

\begin{enumerate}
\item Run the Apriori algorithm on your data and find frequent itemsets as well
  as association rules with high confidence.
\item Try and interpret the association rules generated.
\end{enumerate}

%%%%%%%%%%%%%%%%%%%% Outlier / Anomaly detection
\section{Outlier / Anomaly detection}
\label{sec:detection}
In this part of the exercise you should apply some of the scoring methods for
detecting outliers you learned in Exercise 11.  In particular, you should

\begin{enumerate}
  \item Rank all the observations in terms of the Gaussian Kernel density (using
    leave- one-out), KNN density, KNN average relative density and distance to
    Kth nearest neighbor for some suitable K. (If the scale of each attribute in
    your data are very different it may turn useful to normalize the data prior
    to the analysis).
  \item Discuss whether it seems there may be outliers in your data according to
    the four scoring methods.
\end{enumerate}

%%%%%%%%%%%%%%%%%%%% Results and discussion
\section{Results and discussion}
\label{sec:results_and_discussion}

%%%%%%%%%%%%%%%%%%%% Bibliography %%%%%%%%%%%%%%%%%%%%
\begin{thebibliography}{10}
\bibitem{datadescription} \url{http://archive.ics.uci.edu/ml/datasets/Abalone}
\bibitem{gam} \url{https://en.wikipedia.org/wiki/Generalized_additive_model}
\bibitem{Waugh.thesis} S.~Waugh,''Extending and Benchmark Cascade-Correlation,''
  Thesis, 1997.
\bibitem{Mayukh} H.~Mayukh, ``Age of Abalones using Physical
  Characteristics: A Classification Problem,'' ECE 539 Fall 2010
  Project Report, Department of Electrical and Computer Engineering
  University of Wisconsin-Madison, 2010.
\end{thebibliography}
\end{document}
