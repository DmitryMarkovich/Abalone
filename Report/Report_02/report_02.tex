%%%%%%%%%%%%%%%%%%%% Preamble %%%%%%%%%%%%%%%%%%%%
\documentclass[10pt, paper=a4]{article}
\usepackage{amssymb,amsfonts,amsmath,latexsym,amsthm, mathtools} %mathtext,
\usepackage{booktabs}
\usepackage{multirow}
\usepackage{graphicx}
\usepackage{listings}
\usepackage{chngpage}
\usepackage[font=footnotesize, labelsep=period]{caption}
\usepackage{cite}
\usepackage[scale=0.925]{geometry}
\graphicspath{{images/}}
\usepackage[pdftex,unicode,colorlinks, citecolor=blue,
  filecolor=black, linkcolor=blue, urlcolor=blue]{hyperref}
\usepackage[figure,table]{hypcap}
%%%%%%%%%%%%%%%%%%%% Document %%%%%%%%%%%%%%%%%%%%
\begin{document}
%%%%%%%%%%%%%%%%%%%% Title page %%%%%%%%%%%%%%%%%%%%
\title{Report 02}

\author{ Dmitriy Markovich, Julian Lemos Vinasco}

\date{}

\maketitle

\begin{abstract}
  Objective: The objective of this second report is to apply the
  methods you have learned in the second section of the course on
  ”Supervised learning: Classification and regression” in order to
  solve both a relevant classification and regression problem for your
  data.
\end{abstract}

%%%%%%%%%%%%%%%%%%%% Introduction %%%%%%%%%%%%%%%%%%%%
\section{Regression}
\label{sec:regression}

\subsection{Problem description}
The linear regression objective is to predict the Abalon's age based on 9 given attributes. The attributes considered for linear regression (LR) are presented next: Sex, Length, Diameter, Height, Whole weight (WhlWght), Shucked weight (ShckdWght), Viscera weight (VscrWght) and Shell weight (ShllWght). More details about the data set can be seen in \cite{datadescription}.

\subsection{Linear Regression with forward selection}
Prior to the LR the relation between the dependent variable Age and the independent attibutes were explored by a generalized additive model. Fig.~\ref{fig:gam} shows 

\begin{figure}[h]
  \centering
  \includegraphics[width = 0.99\textwidth]{gam.pdf}
  \caption{Attributes relation with Age}
  \label{fig:gam}
\end{figure}

%% Test images from Dmitriy are depicted on Fig.~\ref{fig:test}.
%% \begin{figure}
%%   \begin{minipage}{0.49\textwidth}
%%     \includegraphics[width = 0.99\textwidth]{download.jpg}
%%   \end{minipage} \hfill
%%   \begin{minipage}{0.49\textwidth}
%%     \includegraphics[width = 0.99\textwidth]{webmd_rf_photo_of_liver_and_hepatitis_virus.jpg}
%%   \end{minipage} \vfill
%%   \caption{Test figures of Hepatitis from Dmitriy.}
%%   \label{fig:test}
%% \end{figure}
%%%%%%%%%%%%%%%%%%%% Classification %%%%%%%%%%%%%%%%%%%%
\section{Classification}
\label{sec:classification}

\begin{enumerate}
\item Explain which classification problem you have chosen to solve.
\item Apply at least three of the following methods: Decision Trees
  (as in Fig.~\ref{fig:decision_tree}), Logistic/Multinomial
  Regression (as in Lst.~\ref{lst:logistic_regression}), K-Nearest
  Neighbors (KNN), Naı̈ve Bayes and Artificial Neural Networks (ANN).
  (Use cross-validation to select relevant parameters in an inner
  cross-validation loop and give in a table the performance results
  for the methods evaluated on the same cross-validation splits on the
  outer cross-validation loop, i.e. you should use two levels of
  cross-validation).
\item For the models you are able to interpret explain how a new data
  observation is classified.  (If you have multiple models fitted,
  (i.e., one for each cross-validation split) either focus on one of
  these fitted models or consider fitting one model for the optimal
  setting of the parameters estimated by cross-validation to all the
  data.)
\item Statistically compare the performance of the two best performing
  models (i.e., use a paired t-test). Compare in addition if the
  performance of your models are better than simply predicting all
  outputs to be the largest class in the training data.
\end{enumerate}


The classification problem for the dataset is to predict the age range
of abalones using their measured physical characteristics.  Age in
years as an attribute can be calculated from the Rings attribute as
Age = 1.5 + Rings.  Age range may be obtained from the Age attributes
by splitting it into a number of groups.  Grouping can be done in a
variety of ways, but it is wise to group the data in a way that the
number of counts in each group is at least on the same scale.  The
distribution of Age attribute and its splitting into 5 groups is
presented in Fig.~\ref{fig:age_grouping}.

\begin{figure}[htbp]
  \centering
  \includegraphics[width = 0.99\textwidth]{age_grouping.pdf}
  \caption{Distribution and grouping of Age attribute.}
  \label{fig:age_grouping}
\end{figure}

Each of the groups in Fig.~\ref{fig:age_grouping} represents a certain
class, that the Age attribute belongs to.  So we can proceed with
predicting the age range or the class of age using other attributes.
The results of applyig a decision tree method are presented in
Fig.~\ref{fig:decision_tree}.


\begin{figure}[htbp]
  \centering
  \includegraphics[width = 0.99\textwidth]{decision_tree.pdf}
  \caption{Decision tree for the class of Age attribute.}
  \label{fig:decision_tree}
\end{figure}

%% \begin{lstlisting}[label = lst:logistic_regression, caption = {Results of logistic regression}]
%%   Call:
%% glm(formula = Class ~ Malaise + Ascites + Bilirubin + Histology, 
%%     family = binomial(link = logit), data = data)

%% Deviance Residuals: 
%%     Min       1Q   Median       3Q      Max  
%% -2.7029   0.2382   0.2888   0.4330   1.8019  

%% Coefficients:
%%             Estimate Std. Error z value Pr(>|z|)   
%% (Intercept)   1.2874     0.8407   1.531  0.12567   
%% Malaise       0.9020     0.5670   1.591  0.11162   
%% Ascites       1.9074     0.6621   2.881  0.00397 **
%% Bilirubin    -0.7838     0.2613  -3.000  0.00270 **
%% Histology    -0.9933     0.6132  -1.620  0.10529   
%% ---
%% Signif. codes:  0 ‘***’ 0.001 ‘**’ 0.01 ‘*’ 0.05 ‘.’ 0.1 ‘ ’ 1

%% (Dispersion parameter for binomial family taken to be 1)

%%     Null deviance: 145.117  on 144  degrees of freedom
%% Residual deviance:  91.903  on 140  degrees of freedom
%%   (10 observations deleted due to missingness)
%% AIC: 101.9

%% Number of Fisher Scoring iterations: 5
%% \end{lstlisting}

%%%%%%%%%%%%%%%%%%%% Results and Discussion %%%%%%%%%%%%%%%%%%%%
\section{Results and Discussion}
\label{sec:results_and_discussion}
If your data has previously been analyzed by regression or
classification in the lit- erature, please report what methods have
been used previously as well as their performance and relate your
results to these previous results.

Notice, if the analysis of your data is too computationally demanding
for choosing parameters in the inner cross-validation loop we suggest
you use the hold-out method instead of K-fold
cross-validation. Furthermore, if analyzing the data by ANN is too
computationally demanding you can consider only analyzing a subset of
your data by ANN.

The report should be 5-10 pages long including figures and tables and
give a precise and coherent account of the results of the regression
and classification methods applied to your data. Please hand in the
report by uploading it as a single, uncompressed .pdf file to
CampusNet no later than {\bf 12 April at 13:00}.  To ensure all group
members get credit for the report, put your names and study numbers on
the front page and ensure you upload the report as a group hand in and
put the name of your dataset on the front page.
%%%%%%%%%%%%%%%%%%%% Bibliography %%%%%%%%%%%%%%%%%%%%
\begin{thebibliography}{99}
	\bibitem{Waugh.thesis} S.~Waugh,''Extending and Benchmark
  	Cascade-Correlation,'' Thesis, 1997.
	\bibitem{Mayukh} H.~Mayukh, ``Age of Abalones using Physical
    Characteristics: A Classification Problem,'' ECE 539 Fall 2010
    Project Report, Department of Electrical and Computer Engineering
    University of Wisconsin-Madison, 2010.
    \bibitem{datadescription} \url{http://archive.ics.uci.edu/ml/datasets/Abalone}
    
\end{thebibliography}
\end{document}
